\chapter{Steuerung des Programmablaufs}
\section{Konstanten - Boolesche Werte}
\begin{tabular}{ll}
\texttt{True}	&wahr\\
\texttt{False}	&falsch\\
\texttt{None}	&undefiniert\\
\end{tabular}
\section{Objekt- vs. Wertevergleich}
\begin{lstlisting}
#Eindeutige ID des Objekts (Referenz)
id(object1)

#Identitaetsvergleich (kein Vergleich des Inhalts, sondern der Referenz)
object1 is object2
object1 is not object2

#Zuweisung auf selbe Referenz
object1 = 1
object2 = 2
object1 = object2
object1 is object2 #True

#Wertvergleich
object2 = 3
object1 == object2 #False

\end{lstlisting}

\section{Boolesche Operatoren}
Diese Operatoren haben die niedrgste Priorität aller bis jetzt vorgestellten Operatoren.\\[0.5em]
\begin{tabular}{ll}
\texttt{or}		&Logisches Oder\\
\texttt{and}	&Logisches Und\\
\texttt{not}	&Logische Negation\\
\end{tabular}
\section{Bedingte Ausführung \texttt{if}}
\begin{lstlisting}
a = 42

#Einfache if-Abfrage
if a == 42:
	print(a)
	
#Mehrere Bedingungen
if a == 0:
	print(''0'')
elif a > 0:
	print(a)
else:
	print(''kleiner'')
\end{lstlisting}
\section{Objektabhängige Ausführung \texttt{with}}
Mit dem \texttt{with}-Statement wird ein Block abhängig von einem Context ausgeführt. Der Context wird durch ein Objekt dargestellt. Es werden Methoden gestellt, die zu Beginn und Ende des Blocks ausgeführt werden.
\begin{lstlisting}
with expression [as variable]:
    with-block
\end{lstlisting}
\section{Bedingter Ausdruck}
Ternärer Operator als Platzersparnis im Gegensatz zu einer vollständigen If-Anweisung.
\begin{lstlisting}
#Binaerer Ausdruck
if True:
   1
else:
   0

#Ternaerer Ausdruck
1 if True else 0


\end{lstlisting}
