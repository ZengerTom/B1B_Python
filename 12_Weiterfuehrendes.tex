\chapter{Weiterführendes}
\section{Typ einer Variablen}
\begin{lstlisting}
liste = [1, 2, 3]
type(liste) == list    #true
\end{lstlisting}
\section{Attribute eines Objektes}
\begin{lstlisting}
#Ausgabe aller Attribute, Methoden des Objektes
liste = [1, 2, 3]
dir(liste)

#Testen aus Existenz
hasattr(liste, '__doc__')
\end{lstlisting}
\section{Standardfunktionen implementieren}
\begin{tabular}{lp{8cm}}
\textbf{Methodenname}		&\textbf{Einsatzzweck}\\
\texttt{\_\_init\_\_(self [, ...])}	&Konstruktor\\
\texttt{\_\_del\_\_(self)}		&Destruktor\\
\texttt{\_\_repr\_\_(self)}		&Liefert Darstellung des Objekts als String\\
\texttt{\_\_str\_\_(self)}		&Liefert String\\
\texttt{\_\_bytes\_\_(self)}		&Byte-String des Objets\\
\texttt{\_\_format\_\_(self, f)}	&ruft \texttt{format() auf}\\
\texttt{\_\_hash\_\_(self)}		&ruft \texttt{hash()} auf, sollte nur zusammen mit \texttt{eq()} implementiert werden\\
\texttt{\_\_bool\_\_(self)}		&bewertet das Ergebnis von \texttt{len()}\\
\end{tabular}
\section{Vergleichsoperatoren}
\begin{tabular}{lp{8cm}}
\textbf{Methodenname}		&\textbf{Einsatzzweck}\\
\texttt{\_\_eq\_\_(self, o)}	&equal\\
\texttt{\_\_gel\_\_(self, o)}		&greater or equal\\
\texttt{\_\_gt\_\_(self, o)}		&greater\\
\texttt{\_\_le\_\_(self, o)}		&less or equal\\
\texttt{\_\_lt\_\_(self, o)}		&less than\\
\texttt{\_\_ne\_\_(self, o)}		&not equal\\
\end{tabular}
\section{Attributzugriif}
\begin{tabular}{lp{8cm}}
\textbf{Methodenname}				&\textbf{Einsatzzweck}\\
\texttt{\_\_getattr\_\_(self, n)}			&Wird aufgerufen, falls Attribut nicht in verfügbaren Namensraum gefunden\\
\texttt{\_\_getattribute\_\_(self, n)}		&Wird zuerst beim Attributzugriff aufgerufen\\
\texttt{\_\_setattr\_\_(self, n, v)}			&Wird aufgerufen, wenn einem Attribut ein Wert zugewiesen wird\\
\texttt{\_\_delattr\_\_(self, n)}			&Wird durch \texttt{del o.n}aufgerufen\\
\texttt{\_\_dir\_\_(self)}				&Aufruf von \texttt{dir()}\\
\end{tabular}


