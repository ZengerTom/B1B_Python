\chapter{Objekttypen}
Alle Datentypen sind Objekte. Mit dem Typ sind Funktionen verknüpft (Objektmethoden).
\begin{lstlisting}
#Methodenaufruf
objekt.methode()
\end{lstlisting}
\section{Zahlen}
\begin{lstlisting}
ganzzahl = 42 #int
langezahl = 43218932549085324908
fliesskomma1 = 3.12 #float
fliesskomma2 = 8.12e+10
komplex = 1+2j #complex
\end{lstlisting}
\textbf{Besonderheiten}
\begin{itemize}
\item Ganze Zahlen können Werte über den gültigen Speicherbereich annehmen
\item Ganze Zahlen haben keine Begrenzung bei der Genauigkeit
\item Angabe von Ganzen Zahlen kann auch binär, oktal oder hexadezimal erfolgen (Präfixe: \texttt{0b}, \texttt{0o}, \texttt{0x})
\end{itemize}
\subsection{Arithmetische Operatoren}
\begin{tabular}{lp{4cm}lp{6cm}}
\texttt{+}	&Addition	&\texttt{abs(x)}	&absoluter Wert\\
\texttt{-}	&Subtraktion	&\texttt{int(x)}	&erzeugt Ganzzahl\\
\texttt{*}	&Multiplikation	&\texttt{float(x)}	&erzeugt Fließkommazahl\\
\texttt{/}	&Division	&\texttt{complex(x)}	&erzeugt komplexe Zahl\\
\texttt{//}	&Ganzzahldivision	&\texttt{divmod(x)}	&Ganzzahl- \& Modulodivision\\
\texttt{\%}	&Modulo	&\texttt{pow(x, y)}	&Potenzieren\\
\texttt{-}	&Negation	&\texttt{x ** y}	&Potenzieren\\
\texttt{+}	&unveränderter Wert	&	&
\end{tabular}
\subsection{Bit-Operatoren}
\begin{tabular}{lp{4cm}lp{6cm}}
\texttt{a $\mid$ b}	&ODER	&\texttt{a $\ll$ b}	&n Bit nach links\\
\texttt{a \^{} b}	&XOR	&\texttt{a $\gg$ b}	&n Bit nach rechts\\
\texttt{a \& b}	&UND	&\texttt{$\sim$a}	&Bitkomplement\\
\end{tabular}
\subsection{Vergleichsoperatoren, Boolsche Operatoren}
\begin{tabular}{lp{4cm}lp{6cm}}
\texttt{a $<$ b}	&kleiner	&\texttt{a $>=$ b}	&größer gleich\\
\texttt{a $<=$ b}	&kleiner gleich	&\texttt{a $==$ b}	&gleich\\
\texttt{a $>$ b}	&größer	&\texttt{a $!=$b}	&ungleich\\
\texttt{AND}	&UND	&\texttt{OR}	&ODER\\
\texttt{NOT}	&NICHT
\end{tabular}
\subsection{Kurzschreibweise}
\begin{lstlisting}
a += b
\end{lstlisting}
\section{Zeichenketten}
Python speichert Zeichenketten als einzelne Zeichen innerhalb eines String-Objekts mit UTF-8. Ein Typ für ein einzelnes Zeichen ist nicht vorhanden, alle Zeichenketten gehören der Klasse \texttt{str} an. Zeichenketten sind unveränderlich, alle Aktionen liefern eine neue Zeichenkette.
\begin{lstlisting}
#Gleichwertig, einfaches und doppeltes Anfuhrungszeichen ohne Escapen
s = 'hello'

s = ''hello''

#Multiline String
s = '''hello
... world
...'''
\end{lstlisting}
\section{Listen}
Eine Liste ist eine geordnete Sammlung von Objekten. Eine Liste kann beliebige Objekte enthalten und ist veränderbar.
\begin{lstlisting}
#leere Liste
liste = []

#Liste mit Lange n
liste = [n]

#direkte Zuweisung
liste = [1, 3.14, ''hello'']
\end{lstlisting}
\section{Tupel}
Tupel sind wie Listen eine geordnete Sammlung von Objekten. Im Gegensatz zur Liste sind Tupel nicht veränderbar
\begin{lstlisting}
#leeres Tupel
tupel = ()

#direkte Zuweisung
tupel = (1, 2, 3)

#Ein-Element-Tupel
tupel = (1,)
\end{lstlisting}
\section{Range}
Ein Bereichstyp zur Erzeugung von unveränderbaren Zahlenlisten. Es können bis zu drei Parameter bei der Erzeugung angegeben werden (Startwert, Endwert, Schrittweite). Standardstartwert ist 0.
\begin{lstlisting}
list(range(1, 10, 2)

r = range(10)
\end{lstlisting}
\section{Operationen auf Sequenztypen}
Gemeinsamer Satz von Operationen und Funktionen für Objekte des Typs Strings, Listen, Tupel und Range.
\subsection{Index-Zugriff}
\begin{itemize}
\item Abfrage erfolgt über Index (Nur bei Liste: Zuweisung)
\item Index beginnt mit 0
\item Letztes Element hat den Index Länge - 1
\item Negativer Index: Es wird ab Ende der Sequenz gezählt
\end{itemize}
\subsection{Slicing}
\begin{lstlisting}
list = [1, 2, 3, 4, 5, 6]

#Extrahieren Teilstuck druch Index, Ende exklusiv
list[start:ende:schrittweite]

#Loeschen eines Listenbereichs
list[start:ende] = []
\end{lstlisting}
\subsection{weitere Funktionen}
\begin{lstlisting}
''hallo'' + ''welt'' 	#''hallo welt''
''hallo'' * 2		#''hallo hallo''
s = ''hallo''
''ll'' in s		#true
''oo'' not in s		#true
len(s)			#5
s.index('a')		#1
min(s)			#Minimum ' '
max(s)			#Maximum 'o'
s.count('s')		#2
\end{lstlisting}
\section{Dictionarys}
Der Typ \texttt{dict} ist eine Sammlung von Schlüssel-Wert-Paaren, wobei der Schlüssel unveränderlich ist. Es existiert keine definierte Reihenfolge, ein Zugriff über einen Index ist somit nicht möglich.
\subsection{Deklaration}
\begin{lstlisting}
d = {}
d = {'key1' : 'wert1', 'key2' : 'wert2', 'key2' : 'wert2', }
\end{lstlisting}
\subsection{Zugriff, Zuweisung, Löschen}
Vor dem Zugriff sollte man überprüfen, ob der Schlüssel enthalten ist (\texttt{in}).
\begin{lstlisting}
'key1' in d

#Zugriff
d['key1']

#Zuweisung
d['key1'] = 42

#Entfernen Key und Value
del d['key1']
\end{lstlisting}
\subsection{Operationen}
\begin{lstlisting}
#Erweitern
d.update(d2)

#Alle Keys bzw. Werte
d.keys()
d.values()

#Alle Paare
d.items()
\end{lstlisting}
Die Methoden liefern ein Objekt welches mit der List-Mehtode in eine Liste umgewandelt werden kann (\texttt{list(d.values()}).
\section{Set und Frozenset}
Veränderbare Menge von unveränderbaren Objekten, welche mathematische Mengenoperationen beherrscht. Kein Index, Sortierung und Slicing.  Zusätzlich arbeitet das Frozenset wie ein Dictionary und kann nicht verändert werden.
\begin{lstlisting}
s = {1, 2, 'a', 3.14}
s.add('=')
s.remove(2)
2 in s
len(s)
\end{lstlisting}
\subsection{Mengenoperationen}
\begin{tabular}{lp{4cm}p{6cm}}
\textbf{Operator}	&\textbf{Methode}	&\textbf{Funktion}\\
$<=$		&issubset			&Untermenge\\
$>=$		&issuperset		&Obermenge\\
$\mid$	&union			&Vereinigungsmenge\\
\&		&intersection		&Schnittmenge\\
$-$		&difference		&Differenzmenge\\
\texttt{ \^{} }	&symmetric\_difference	&Differenz mit einmaligen Werten
\end{tabular}
\section{Erzeugung von Listen, Sets und Dictionarys}
Außer der expliziten Deklaration steht den Objekten der Typen list, set  und dict die Erzeugung durch eine Schleife zur Verfügung (``List Comprehension'' bzw. ``Display'').
\begin{lstlisting}
#Explizite Erzeugung
l = ['a', 'b', 'c']
s = {1, 2, 3}
d = {'a' : 1, 'b' : 2}

#Allgemeine Form
<Ausdruck> for <Schleifenvariable> in <Sequenz> <Test>

#List
l = [x for x in rage(10, 15)]
s = {x / 2 for x in range(5) if x % 2 == 0}
d = {x : y for x, y in enumerate(l)}
\end{lstlisting}

