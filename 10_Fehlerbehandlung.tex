\chapter{Fehlerbehandlung}
\section{Fehler abfangen \texttt{try ... except}}
\begin{lstlisting}
try:
	#do sth
	
#einen Fehler abfangen
except KeyboardInterrupt as err:
    print(''{}''.format(error))
	
#Verschiedene Fehler abfangen
except (TypeError, SyntaxError):

#alle Fehler abfangen
except:
    pass

#nach try weiter ohne Fehler
else:
    #Anweisungen, die keinen Fehler ausloesen koennen
    
finally:
    #Vor Beendigung ausfuehren
\end{lstlisting}
\section{Exception auslösen}
Eine Exception kann mit dem Schlüsselwort \texttt{raise} ausgelöst werden. Nach dem Schlüsselwort wird die gewünschte Exception und ggfls. Parameter in Klammern angegeben.
