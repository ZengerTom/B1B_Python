\chapter{Schleifen}
\section{Zählschleife \texttt{for}}
Durchläuft alle Elemente eines Objekts. Bei einem Dictionary wird der key zurückgeliefert. Die Rückgabe der Schlüsselwerte erfolgt nicht in der Reihenfolge, in der das Dictionary initialisiert wurde (erzwingbar mit \texttt{sorted()}).
\begin{lstlisting}
for a in ''Hello'':
	print(a)		#H e l l o

#Dictionary Zugriff Mehrfachzuweisung
d = {'1' : 4, '2' : 8, '3' : 16}
for a, b in d.items():
	print(a, b)
	
#Range und (optional) Schleifenende
for n in range(3):
	print(n)
else:
	print(''Ende'')
\end{lstlisting}
Der Endteil kann mit der \texttt{break}-Anweisung unterbunden werden.
\section{Bedingte Schleife \texttt{while}}
Bedingung wird vor jeder Ausführung geprüft.
\begin{lstlisting}
i = 0
while i < 10:
	i += 1
#optional
else:
	print(''Ende'')
\end{lstlisting}
\section{Unterbrechung und Neustart}
\begin{lstlisting}
#Abbruch der Schleife
for n in range(10):
	if n == 2:
	   break
	print(n)
	
#neue Iteration
for n in range(10):
	if n % 2 == 0:
		continue
	print(n)
\end{lstlisting}