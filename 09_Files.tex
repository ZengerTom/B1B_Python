\chapter{Ein- und Ausgabe}
\section{Eingabe \texttt{input()}}
Einfachste Form mit Einlesen von der Tastatur mit dem Befehl \texttt{input()}. Beendet wird das Einlesen durch:
\begin{itemize}
\item Return oder Enter
\item \texttt{Ctrl + C}
\item \texttt{Ctrl + D}
\end{itemize}
\begin{lstlisting}
s = input('Eingabe: \n')
\end{lstlisting}
\section{Ausgabe \texttt{print()}}
Gegenstück zu \texttt{input()}. Nimmt beliebig viele Parameter entgegen und fügt am Ende das Newline Zeichen ein.
\subsection{Formatierung}
Ausgabeformatierung mithilfe der Methode \texttt{format()}. Platzhalter werden durch die übergebenen Werte/Variablen ersetzt. Variablen können auch über ihren Index oder Namen angesprochen werden\\[0.5em]
\begin{tabular}{ll}
\textbf{Platzhalter}	&\textbf{Funktion}\\
\{\}			&Universeller Platzhalter\\
\{ name \}		&Variable mit Namen \texttt{name} hier einsetzen\\
\{ :5 \}		&Feld 5 Zeichen breit formatieren\\
\{ n:5 \}		&Variable 5 Zeichen breit einfügen\\
\{ :10b \}		&Feld 10 Zeichen breit und Binärdarstellung\\
d			&Dezimaldarstellung\\
f			&Fließkommazahl (6 Nachkommastellen)\\
.2f			&2 Nachkommastellen\\
b			&Binärdarstellung\\
x,X 			&Hexadezimaldarstellung mit kleinen oder großen Buchstaben\\
!a 			&Funktion \texttt{ascii() auf Variable anwenden}\\
!s			&Einsatz von \texttt{str()}\\
!r			&Einsatz von \texttt{repr()}\\
\end{tabular}
\\[0.5em]
\begin{lstlisting}
'p2 {1:10} p1 {0:04X} k2 {key2} k1 {key1}'.format(42, 3.15, key1='a', key2='b')
#p2       3.15 p1 002A k2 b k1 a
\end{lstlisting}
\subsubsection{Formatierung mit \texttt{str}-Methoden}
\begin{tabular}{lp{9,5cm}}
\textbf{Methode}				&\textbf{Beschreibung}\\
\texttt{ljust(w)}					&Text links ausgeben, Zeichenkette der Länge w\\
\texttt{rjust(w)}					&Text rechts ausgeben , Zeichenkette der Länge w\\
\texttt{zfill(w)}					&Zeichenkette der Breite w links mit 0 füllen\\
\texttt{center(w[, fillchar])}		&Text in de Mitte einer Zeichenkette der Breite w positionieren und ggf. das Füllzeichen verwenden\\
\texttt{captialize()}				&Erste Zeichen in Großbuchstaben, der Rest klein\\
\texttt{lower()}					&Alle Zeichen in Kleinbuchstaben\\
\texttt{swapcase()}				&Klein- gegen Großbuchstaben und umgekehrt\\
\texttt{title()}					&Erster Buchstabe jedes Wortes in Großbuchstaben\\			
\end{tabular}
\subsection{Formatierung mit Formatstrings}
Der Formatstring ist eine Zeichenkette mit Steuerzeichen als Platzhalter. Die Variablen werden nach einem \verb+%+ angegeben.\\[0.5em]
\begin{tabular}{ll}
\textbf{Platzhalter}	&\textbf{Ausgabe}\\
\verb+%s+		&Zeichenkette\\
\verb+%d+		&Ganzzahl\\
\verb+%x+		&Ganzzahl Hexadezimal Klein\\
\verb+%X+		&Ganzzahl Hexadezimal Groß\\
\verb+%o+		&Ganzzahl Oktal\\
\verb+%f+			&Fließkommazahl\\
\verb+%2d+		&Anzahl Zeichen\\
\verb+%02s+		&Anzahl Zeichen mit ggfls. führenden Nullen\\
\verb+%.2f+		&Beschränkung Nachkommastellen\\
\verb+%5.2f+		&Gesamtbreite mit Nachkommastellen\\
\verb+%10s+		&Breite 10 Zeichen rechtsbündig\\
\verb+%-10s+		&Breite 10 Zeichen linksbündig\\
\end{tabular}
\subsection{Ausgabe auf \texttt{stdout} oder \texttt{stderr}}
Ausgabe mit der \texttt{write()} Methode auf den Standard Ausgabekanal. Der Import der Kanäle erfolgt mit \texttt{sys}.
\section{Dateien}
\subsection{Arbeiten mit Dateien}
Für Textdateien reicht die Angabe des Namens. Bei anderen Dateitypen muss der Modus mit angegeben werden.\\[0.2em]
\begin{tabular}{ll}
\textbf{Parameter}	&\textbf{Funktion}\\
r	&Lesen (Default)\\
w	&Schreiben (Löscht evtl. vorhandene Datei)\\
a	&am Ende der Datei anhängen\\
r+	&Lesen \& Schreiben\\
b	&angehängt an anderen Modi, öffnet Datei im Binärmodus\\
\end{tabular}
\begin{lstlisting}
fd = open ('foo')
data = fd.read()
fd.close
\end{lstlisting}
\subsection{Methoden von File-Objekten}
\begin{tabular}{ll}
\texttt{read(n)}		&Liest n Bytes aus der Datei\\
\texttt{readline()}	&Liest eine Zeile inklusive des Zeilenendezeichens\\
\texttt{readlines()}	&Liefert eine Liste aller Zeilen in der Datei\\
\texttt{write(s)}		&Schreibt Zeichenkette s in Datei\\
\texttt{seek(p)}		&Setzt Position des Zeigers auf Position p\\
\texttt{tell()}		&Liefert die Position des Zeigers\\
\end{tabular}
\subsection{Daten in Objekte ausgeben}
Das Modul \texttt{io} bietet eine String-Klasse, die wie ein File-Objekt behandelt werden kann: \texttt{StringIO}. StringIO verhält sich wie eine Datei.
\subsection{Fehlerbehandlung bei der Arbeit mit Dateien}
\begin{lstlisting}
try:
    file = open{'robots.txt'}
except (FileNotFoundError, PermissionError) as err:
    print(''Fehler {}''.format(err))
    
try:
    data = file.read()
except: ValueError
    pass
finally:
    file.close()
\end{lstlisting}
Äquivalent zu obigen Code mit Context Manager:\\[0.5em]
\begin{lstlisting}
with open('robots.txt', 'r') as file:
    f.read()
\end{lstlisting}

