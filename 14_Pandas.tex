\chapter{Pandas - Python and data analysis}
Pandas ist ein Python-Modul zur Daten Manipulation und Analyse. Die Bibliothek baut auf Numpy auf. Scipy und Matplotlib stellen eine sinnvolle Ergänzung zu Pandas da.
\section{Datenstrukturen}
\subsection{Series}
Eine Series ist ähnlich dem eindimensionalen Array. Ein Array fungiert hierbei als Index (Label), in einem weiteren Array werden die Daten gespeichert (Werte).
\begin{lstlisting}
import pandas as pd
import numpy as np

s = pd.Series([10, 20, 30, 40, 50])

#Indizies umbennen
s.index = ['START', 'QUARTER', 'HALF', 'THREE-QUARTER', 'ENDE']

#Series mit bestimmten Index erstellen
ind = ['START', 'QUARTER', 'HALF', 'THREE-QUARTER', 'ENDE']
t = pd.Series([100, 200, 300, 400, 500], index=t )

#Zugriff Index
s['START']

#Informationen zu Series
s.describe()
\end{lstlisting}
\subsection{Data Frame}
Ein Data Frame besitzt einen Zeilen- und einen Spaltenindex. Es ist vergleichbar mit einem Dictionary aus Series mit einem normalen Index. Jede Series wird über einen Index (Name de Spalte) angesprochen. Bei den meisten Operationen wird eine Kopie des Data Frames erzeugt und nicht der Data Frame selbst geändert.
\begin{lstlisting}
import pandas as pd
import numpy as np

years = range(2014, 2018)
a = pd.Series([10.1, 10.2, 10.3, 10.4], index=years)
b = pd.Series([20.1, 20.2, 20.3, 20.4], index=years)
c = pd.Series([30.1, 30.2, 30.3, 30.4], index=years)

#Series aneinanderreihen (hintereinander)
d = pd.concat([a, b, c])

#Data Frame erzeugen aus Series
df = pd.concat([a, b, c], axis=1)

#Spaltennamen
df.columns				#Ausgabe der Range
shops_df.columns.values		#Ausgabe der Werte

df.columns = ['a', 'b', 'c']		#Spaltennamen umbenennen

#Spaltennamen vor Konkatenierung angeben
a.name = 'a_new'
b.name = 'b_new'
c.name = 'c_new'

#Zugriff auf Spalte durch Indizierung
df['a','b']
print(df.a)

#Zugriff auf bestimmte Reihen
df[0:2]
df.loc[2017]		#liefert Ergebnis als Series
df.loc[[2016, 2017]]	#liefert Ergebnis als Data Frame

#Zugriff auf Reihen und Spalten
df[0:2, ['a', 'b']]

#Bedingter Zugriff
df[df.a > 10]

#Spalten umbenennen
df2 = df.rename(columns={'a'='alt'})

#Reihen umsortieren
df.reindex(index=[2017, 2016, 2015, 2014])
\end{lstlisting}
\subsubsection{Fehlende Werte}
\begin{lstlisting}
#Reihen mit fehlende Werte entfernen
df.dropna()

#Fehlende werte ersetzen
df.fillna()
\end{lstlisting}
\section{Map and Apply}
\subsection{Map}
\begin{lstlisting}
#Elementweise auf Serie
df['str'].dropna().map(lambda x : 'map_' + x)
\end{lstlisting}
\subsection{Apply}
\begin{lstlisting}
#Arbeitet auf Spalten-/Reihenbasis (axis=0 rows, axis=1 cols)
f = lambda x : x +1
df.apply(f, axis=0)
\end{lstlisting}
\subsection{Applymap}
\begin{lstlisting}
#Elementweise
f = lambda x : x +1
df.applymap(f)
\end{lstlisting}
\section{Vectorized Operations}
\begin{lstlisting}
df = pd.DataFrame(data={"A":[1,2], "B":[4,5]})
df["C"] = df["A"]+df["B"]
\end{lstlisting}
\section{Groupings}
\begin{lstlisting}
#erzeugt Objekt
ef = df.groupby('Spaltenname')

#Anzeigen pro Gruppierungselement
ef.size()

#Funktionen sind anwendbar
ef.A.sum()
\end{lstlisting}
\section{Pandas Stats}
\begin{lstlisting}
#Statistik zum Data frame
df.describe()

#Kovarianz
df.cov()

#Korrelation
df.corr()
\end{lstlisting}
\section{Merge und Join}
\subsection{Left Join}
nur Keys des linken Frame
\begin{lstlisting}
pd.merge(left_frame, right_frame, on='key', how='left')
\end{lstlisting}
\subsection{Right Join}
nur Keys des rechten Frame
\begin{lstlisting}
pd.merge(left_frame, right_frame, on='key', how='right')
\end{lstlisting}
\subsection{Outer Join}
Full Join, alle Keys, falls Key in einem Frame nicht vorhanden NaN
\begin{lstlisting}
pd.merge(left_frame, right_frame, on='key', how='left')
\end{lstlisting}
\subsection{Inner Join}
Intersection beider Frames (nur Keys, die in beiden Frames vorhanden)
\begin{lstlisting}
pd.merge(left_frame, right_frame, on='key', how='left')
\end{lstlisting}














